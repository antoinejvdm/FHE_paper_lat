\documentclass[master=mcs]{kulemt-tex/kulemt}
\setup{
  title={PIRANA implementation using GBFV},
  author={Ing. Antoine Janssens van der Maelen},
  promotor={Prof.\,Dr. Ir. F. Vercauteren},
  assessor={placeholder},
  assistant={Dr. Ir. R.~Geelen \and Dr. J.~Kang \and Ir. J.~Spiessens},}

% Remove the "%" on the next line for generating the cover page
% \setup{coverpageonly}
% Remove the "%" before the next "\setup" to generate only the first pages
% (e.g., if you are a Word user).
% \setup{frontpagesonly}

% If you want to include other LaTeX packages, do it here. 

% Finally the hyperref package is used for pdf files.
% This can be commented out for printed versions.
\usepackage[pdfusetitle,colorlinks,plainpages=false]{hyperref}
\usepackage{amsmath}
\usepackage{caption}
\usepackage{amssymb} 
\usepackage{enumitem}
\usepackage{msc} % Message Sequence Charts to make the protocle diagrams
\usepackage{tikz} % To draw figures
\usepackage{multirow} % For multirow in tables
% \usepackage{adjustbox} % To resize tables
\usepackage{amsthm} % For definition and theorems
\usepackage{algorithm} % For pseudocode
\usepackage{algpseudocode} % For pseudocode

%%%%%%%
% The lipsum package is used to generate random text.
% You never need this in a real master's thesis text!
\IfFileExists{lipsum.sty}%
 {\usepackage{lipsum}\SetLipsumDefault{11-13}}%
 {\newcommand{\lipsum}[1][11-13]{\par And some text: lipsum ##1.\par}}
%%%%%%%

%\includeonly{chapter-n}
\begin{document}

\begin{preface}
  I would like to thank XXX.
\end{preface}

\tableofcontents*

\begin{abstract}
  The \texttt{abstract} environment contains a more extensive overview of
  the work. But it should be limited to one page.

\end{abstract}

% A list of figures and tables is optional
%\listoffigures
%\listoftables
% If you only have a few figures and tables you can use the following instead
\listoffiguresandtables
% The list of symbols is also optional.
% This list must be created manually, e.g., as follows:
\chapter{List of Abbreviations and Symbols}
\section*{Abbreviations}
\begin{flushleft}
  \renewcommand{\arraystretch}{1.1}
  \begin{tabularx}{\textwidth}{@{}p{16mm}X@{}}
    HE     & Homomorphic encryption \\
    PHE    & Partially homomorphic encryption \\
    SHE    & Somewhat homomorphic encryption \\
    FHE    & Fully homomorphic encryption \\
    RLWE   & Ring learning with errors \\
    LWE    & Learning with errors \\
    BFV    & Brakerski-Fan-Vercauteren scheme \\
    CLPX   & Chen-Laine-Player-Xia \\
    GBFV   & Generalized Brakerski-Fan-Vercauteren scheme \\
    SIMD   & Single instruction multiple data \\
    PIR    & Private information retrieval \\
    NTT    & Number theoretic transform \\
  \end{tabularx}
\end{flushleft}

\section*{Symbols}
\begin{flushleft}
  \renewcommand{\arraystretch}{1.1}
  \begin{tabularx}{\textwidth}{@{}p{18mm}X@{}}
    $\Phi_m(x)$  & $m$-th cyclotomic polynomial (degree $\varphi(m)$) \\
    $\varphi(m)$ & Euler's totient function \\
    $\omega_m$   & Primitive $m$-th root of unity \\
    $\mathbb{Z}_m^\times$ & Units modulo $m$ (indices in $\Phi_m$ product) \\
    $\mathcal{R}$ & Cyclotomic ring $\mathbb{Z}[x]/(\Phi_m(x))$ \\
    $t,\; t(x)$  & Plaintext modulus: integer (BFV) or polynomial (CLPX/GBFV) \\
    $q$           & Ciphertext modulus \\
    $\Delta$     & Scaling factor $q/t$ (or $q/t(x)$) \\
    $m$           & Codeword length in PIRANA; also cyclotomic index when clear \\
    $k$           & Hamming weight of a codeword (PIRANA) \\
    $r$           & Slots per ciphertext (rows in PIR matrix) \\
    $c$           & Database columns $= n/r$ (PIRANA) \\
    $a_i$         & Uniform LWE/RLWE sample coefficient in $\mathbb{Z}_q$ \\
    $s_i,\; s$   & Secret key coefficient / polynomial \\
    $e$           & Error term sampled from $\chi_{\text{err}}$ \\
    $m$       & Message polynomial/plaintext element \\
    $\mathbf{ct}$ & Ciphertext pair $(c_0, c_1) \in \mathcal{R}_q^2$ \\
    $p_{\text{mod}}$ & Prime used to set plaintext modulus in implementations \\
    $\tau(x)$    & Factor of $t(x)$ used to define SIMD slot decomposition \\
    $b$           & LWE second component $b = \mathbf{a}\cdot\mathbf{s} + e + \Delta m$ \\
  \end{tabularx}
\end{flushleft}


% Now comes the main text
\mainmatter

\chapter{Introduction}
\label{cha:intro}

\section{Privacy-enhancing/preserving technologies}

In a world where AI becomes data-driven and in a world where people are getting more aware of how sensitive their data can be, protection of data is crucial. Some legislation sets up a framework to protect data, such as the General Data Protection Regulation (GDPR) for personal data. This requires companies and individuals to implement data protection technologies to ensure data privacy, security and compliance to the legislation. 

So far, encryption technologies are shown to be effective to protect data at rest and in transit. However, when data is used for computation, preserving privacy becomes more complex. To achieve this, several privacy-enhancing or privacy-preserving technologies (PETs) are available. PETs are defined by the European Union Agency for Cybersecurity (ENISA) as "a coherent system of information and communication technology (ICT) measures that protect privacy by eliminating or reducing personal data or by preventing unnecessary and/or undesired processing of personal data, all without losing the functionality of the information system" \cite{EDPSGlossary}. 

The least secure way to handle data used in computation is to compute in clear, on an unprotected
device, since sensitive data can be seen by the computing party. If this party is malicious or if the
data gets leaked, a malicious party could exploit it. To mitigate this risk, several different PETs can be used.
\subsection{Confidential computing}
A computer is built on different layers, going from the hardware layer at the bottom to the application
layer on top. Traditionally privacy technologies aimed to defend the bottom layers, such as the
operating system. However, with the widespread adoption of hosting and cloud services, there is a
switch to protecting the application from the lower layers. In a normal use case, the application has
to trust the layers underneath, such as the OS kernel and hypervisor layer. If an attacker or a bug is
dissimilated in those lower layers, it is possible to attack the application. Therefore, enclaves are used
in confidential computing, creating a trusted execution environment (TEE). Memory inside an enclave
is encrypted and decrypted on the fly, inside the isolated enclave, only code running in the enclave is
allowed to access the data. So, by essentially only trusting the enclave and the CPU, data in use is
protected from the outside of the enclave. Even privileged software, for instance the operating system,
cannot access the memory directly.
The susceptibility to side-channel attacks limits the use of enclaves; for example information can be
revealed from the way the TEE interacts with other parts of the system, thus potentially revealing
what is happening inside the enclave \cite{EnclavesSCA2015}. A second limitation when using confidential computing is the
fact one is trusting the cloud service provider or the specific computer system to properly set up and use the enclave, but this party can be untrustworthy. To mitigate this risk, attestation mechanisms
can be employed to verify the trustworthiness of the provider or system.

\subsection{Federated computation}
Another way to protect data while being processed is to perform federated computation. Firstly, a
local computation is performed which will hide information about the local database. Secondly, the
different parties who did the local computation share their results with a central co¨ordinator. A second
computation is performed on all data collected from the different parties, yielding the desired final
output, without any party having to reveal its raw data.

This approach however has some limitations. Federated computation is a relatively easy and efficient
way to operate, but can become challenging when one wants to perform non-linear operations on the
data that require intermediate sharing\footnote{Exchange of partial results or intermediate results during multi-step computations.} (e.g. high-order moments). An attacker can retrieve some
information from the parties, as the first computation result (local computation by each party) is sent
in clear to the co¨ordinator. Also, an attacker can use a combination of engineered queries to get some
information about a certain subgroup of data or users (differential attack). To prevent this, one can
add noise to the result of the local computation, to obtain differential privacy. However, adding too
much noise will make the data useless. Also, outliers will increase the noise significantly which can
detoriate the accuracy of the results. So, setting the right noise parameters is tricky and requires a
certain amount of expertise.

\subsection{Multi-party computation}
Multi-party computation uses multiple computing parties, with no authority party. There is no computation
on the original data, but data is first split up into shares and these shares are then distributed
between the computing parties. These parties then perform computations using the shares they possess.
The results of these computations are then combined together to find the result of the desired
computation, on the original data.

When one wants to compute non-linear functions, however, MPC becomes more challenging. To perform
the computation there is a need for interaction between the parties, who need to exchange their
shares between each other. This leaks information about the data. To solve this, for example, two random
numbers can be chosen. These random numbers are used to mask the share, while reconstructing
of the answer remains possible \cite{DeCoste2018}. A risk of MPC is collusion. If some of the parties collude, secret
sharing amongst them can make it possible to reconstruct the original data. Additionally, MPC can
bring along high communication overhead, leading to scalability issues.

\subsection{Homomorphic encryption}
Homomorphic encryption allows computation on encrypted data without decrypting. No party performing computations has access to the plaintext data, these data remain encrypted. Partially homomorphic encryption (PHE) is a type of HE that only supports homomorphic multiplication or addition, but not both. Full HE (FHE) and leveled HE (LHE) support both multiplications and additions, but with LHE only up to a limited computation depth. To enhance security, noise is added to the encrypted data when using HE, in the least significant bits as illustrated in Figure \ref{fig:ciphertext_HE}. However, when performing computations the noise can grow beyond the noise padding bits, eventually corrupting the data in LHE. To mitigate this, bootstrapping can be performed in FHE to reduce the amount of noise, thus allowing more computations to be done whilst maintaining data correctness. 

\begin{figure}
    \centering
    \includegraphics[width=0.5\textwidth]{fig/Ciphertext_HE.png} 
    \captionof{figure}{Ciphertext HE \cite{ilaria_chillotti_zama_tfhe_2022}}
    \label{fig:ciphertext_HE}
\end{figure}

FHE is however computationally intensive (thus slow) for large and unstructured data. It also requires
specialized expertise to implement. To address this, some organisations are supporting adoption (e.g.
Google has released an open source compiler for FHE) \cite{Cowan2021}.

\subsection{Comparison of PETs}
As already mentioned, each PET has its strengths and flaws. Also, the suitability of a PET depends on the envisaged application. 

With the rise of cloud computing, users realise their data in the cloud is at risk. As a result
computing in clear, without any protection or encryption, will phase out. Federated computation
seems useful when data is naturally distributed, for instance when training an AI model with data
collected from different user devices. An important vulnerability however is the fact that data privacy
is not guaranteed by itself, the exchanged model parameters can still leak information about the
underlying data. Therefore, federated computation should be combined with other PETs to enhance the security of the underlying data.

Multi-party computation is a good option to maintain privacy of information while computing, but
inherently needs multiple devices each having a different authority. In MPC, sensitive information
can be restored when parties collude. Although this risk can be lowered when devices are distributed
under different authorities, this makes the implementation of MPC challenging, as finding nodes that
will never collude is a difficult (impossible?) task. Therefore, there is still some risk the sensitive
information leaks.

Confidential computing is still evolving and manufacturers are using different approaches to implement
enclaves, this technique offers a high level of protection by keeping data and
code secure in an enclave. The enclave is made on a single machine, there is no need for multiple
devices (in contrast to MPC, federated computation). Confidential computing already found ground in multiple applications, such as Nitro Enclaves at AWS and Intel SGX CPU's. 

FHE offers strong advantages when compared to other PETs. If an implementation of FHE can be proven, we can be (mathematically) sure the data can not be decrypted while being processed. Also, less communication is needed during computation when compared to MPC (multi-party computation) and it has a better track record in terms of security vulnerability when compared to TEE (trusted execution environment). \cite{j_bouman_comparison_nodate}

On the other hand, there are some disadvantages too. Like some other PETS, FHE requires specialized expertise to implement. But, most importantly, FHE is computationally intensive (thus slow) for large and unstructured data. According to Ulf Mattsson, general FHE
processing is 1,000 to 1,000,000 times slower than equivalent plaintext operations. \cite{corporation_security_2025}.


Enhancing the speed of FHE is an attractive research topic, as it would make FHE more suitable for real-world applications. Therefore, in this thesis, we focus on improving the performance of a private information retrieval (PIR) scheme based on FHE. 


However, seeing every PET independent would be a mistake. For instance, when creating a FHE blockchain
network, there is a need for one key for the whole network. Who holds the decryption key will define
the security level, given it to one party is insecure. MPC could be used as a means to distribute the
key to all nodes of the blockchain, thus making the blockchain network more secure. Thus, combining
PETs could enhance security for certain applications\footnote{The blockchain is made (very) secure by FHE, but by distributing the key using MPC, the security is shifted to the
MPC.}.

\section{Thesis outline}
\chapter{Theoretical background}
\label{cha:2}
xxxx

\section{Homomorphic encryption}

So far, encryption technologies are shown to be effective to protect stored and in transit data. However, when data is used for computation, preserving privacy becomes more complex. To achieve this, several privacy-enhancing technologies (PETs) are available. One type of PET is homomorphic encryption (HE), which allows computation on encrypted data without decrypting. No party performing computations has access to the plaintext data, these data remains encrypted. Partially homomorphic encryption (PHE) is a type of HE that only supports homomorphic multiplication or addition, but not both. Full HE (FHE) and Somewhat HE (SHE) support both multiplications and additions, but with SHE only up to a limited computation depth. To enhance security, noise is added to the encrypted data when using HE, in the least significant bits as illustrated in figure \ref{fig:ciphertext_HE}. However, when performing computations the noise can grow beyond the noise padding bits, eventually corrupting the data in SHE. To mitigate this, bootstrapping can be performed in FHE to reduce the amount of noise, thus allowing more computations to be done whilst maintaining data integrity. 

\begin{center}
    \includegraphics[width=0.5\textwidth]{fig/Ciphertext_HE.png} 
    \captionof{figure}{Ciphertext HE \cite{ilaria_chillotti_zama_tfhe_2022}}
    \label{fig:ciphertext_HE}
\end{center}

FHE offers strong advantages when compared to other PETs. For instance, less communication is needed during computation when compared to MPC (multi-party computation) and it has a better track record in terms of security vulnerability when compared to TEE (trusted execution environment). \cite{j_bouman_comparison_nodate}

On the other hand, there are some disadvantages too. FHE requires requires specialized expertise to implement. But, most importantly, FHE is computationally intensive (thus slow) for large and unstructured data. According to Ulf Mattsson, general FHE
processing is 1.000 to 1.000.000 times slower than equivalent plaintext operations. \cite{corporation_security_2025} 

Whilst performing a large number of (complex) operations, noise added to the HE cyphertext will grow and overwrite data. To avoid this, two methods are used: using big integers and bootstrapping. By using big integers, enough space is provided for the noise to grow for the full computation - the so called leveled schemes.  \footnote{BFV and CKKS are often implemented without bootstrapping, as a leveled scheme, but are bootstrappable.}
To have more computation depth possible, bootstrapping operations are performed to reduce the amount of noise in between chains of computations. One should note bootstrapping is computationally and memory-intensive.  

FHE schemes can be divided into multiple generations, depending on the type of bootstrapping techniques. \cite{robin_bootstraping_BGV_BFV}
First generation schemes include schemes like the Gen09 bootstrapping technique, described in 2009, which is illustrated in figure \ref{fig:boot09}. FHE-encrypted data are FHE-encrypted a second time, with a lower level of noise compared to the initial encryption. Then, a bootstrapping key is sent to the computing node, which is the secret key of the initial encryption, encrypted with the public key of the second encryption. Decryption with this bootstrapping key removes the first encryption layer, and one ends up with data solely encrypted via the second FHE-scheme, with a lower level of noise. This type of scheme is no longer used in practical implementations.

\begin{center}
    \includegraphics[width=0.65\textwidth]{fig/Bootstrapping.png} 
    \captionof{figure}{Gen09 bootstrapping \cite{ilaria_chillotti_zama_tfhe_2022}}
    \label{fig:boot09}
\end{center}

Second generation schemes are defined by having a slow and complex bootstrapping. However, bootstrapping cost is compensated by the use of SIMD (Single Instruction Multiple Data) operations, which will distribute this cost over many slot, so many parallel computations. Examples are BGV, BFV and CKKS. 
Third generation schemes are characterized by a very simple and fast bootstrapping procedure. These exhibit lower circuit complexity, faster execution times, and less noise growth when compared to second generation. On the contrary, they will not offer SIMD slots for parallel processing. Examples include torus FHE (TFHE). 

Next to these generations, some \textcolor{red}{FHE} leveled homomorphic encryption schemes where no bootstrapping technique is known for the moment. The CLPX scheme fro Chen et al. \cite{Chen_CLPX_paper} is an example, where the parameters of the scheme are set as such level to allow deep circuit evaluation before noise corrupts the result.

\subsection{General-BFV}
The BFV (Brakerski–Fan–Vercauteren) scheme is 

\subsubsection{Cyclotomic rings}
BFV is built on the RLWE problem (ring learning with errors), which is a hardness problem used in cryptography. We define the m-th cyclotomic polynomial as follows: 
\[
\Phi_m(x) = \prod_{j \in \mathbb{Z}_m^\times} \bigl(x - \omega_m^j\bigr)
\]

\begin{itemize}
    \item $w_m$ is the primitive $m$-th root of unity, $\in \mathbb{C}$ where $m\ge 1$
    \item $\mathbb{Z}_m^\times$ is the unity group of integer modulo $m$.
\end{itemize}

The degree of the cyclotomic polynomial is equal to $\varphi(m)$, the result of the Euler's totient function of m. Although the cyclotomic polynomial have complex roots, it has been proven that the coefficients are integer numbers and the polynomials are monic (leading coefficient is 1) and irreducible. The RLWE problem is then defined over the ring $\mathcal{R} = Z[x]/(\phi_m(x)$. This ring is a subring of the cyclotomic number field $\mathbb{Q}[x]/(\Phi_m(x))$.

\subsubsection{R-LWE}
\textbf{LWE:} The ciphertext is constituted of two parts: uniform random numbers $a_{i}$ and $b$, where $b$ is the sum of the multiplication of the uniform random numbers $a_{i}$ with the secret key $s_{i}$, some Gaussian distributed noise $e$ and the message $m$, normalized by a delta coefficient. The corresponding ciphertext can be represented on a ring, all the possible values of m are put on a ring, at a spacing delta from each other. To decrypt, the decryption formula \eqref{eq:decrypt_LWE} is used, which is equivalent to rounding the value on the ring (which corresponds to delta times the message plus the error) to the closest possible value for m. If the error becomes too large, the value will round to the wrong message value, so returning a faulty message. 

% One-line RLWE ciphertext definition
\begin{equation}
    \text{Encryption:} \quad ct = (a_0, \dots, a_{n-1}, b) \quad where \quad 
b = \sum_{i=0}^{n-1} a_i s_i + e + \Delta m
\label{eq:encrypt_LWE}
\end{equation}
\begin{equation}
\text{Decryption:} \quad
m \approx \frac{b - \mathbf{a} \cdot \mathbf{s}}{\Delta}
\label{eq:decrypt_LWE}
\end{equation}

\begin{itemize}[noitemsep, label={}]
  \item $a_i \in \mathbb{Z}_q$ are chosen uniformly at random
  \item $s_i \in \mathbb{Z}_q$ are the secret key coefficients
  \item $e \in \mathbb{Z}_q$ is a small error term (typically Gaussian)
  \item $m$ is the message encoded in the ring
  \item $\Delta$ is the message scaling factor
  \item $b \in \mathbb{Z}_q$ is the second component of the ciphertext
\end{itemize}
This scheme already allows to do some operations on the ciphertext: we can add two ciphertexts and perform multiplications of the ciphertext with non-encrypted integers.\footnote{An operation on ciphertexts will, in FHE, correspond to an operation on plaintexts.}


\textbf{RLWE:} Ring LWE is similar to LWE but it will, when constructing the ciphertext, use polynomial modulo’s instead (for the message, secret key, uniform random numbers and error). When encrypting, we will normalize the message and add a Gaussian error, like in LWE. For every coefficient of the polynomial, the value of delta m plus the error will again be represented on a ring. Rounding will give the polynomial coefficients of the message m. 
The RLWE scheme allows to perform additions between ciphertexts and multiplication with non-encrypted constant polynomial functions.

The RLWE problem is based on the RLWE distribution for integers $q\ge2$ and a secret s sampled from $\chi_{key}$. The decision RLWE problem is, given many plaintext-ciphertext samples, to decide whether the samples come from a uniform random distribution or from the RLWE distribution. When solving the search RLWE problem, many plaintext-ciphertexts are given from the RLWE distribution, and one needs to find the secret s. Both variants are supposed to be hard. 

\subsubsection{BFV, CLPX and SIMD}
In BFV by Kim et al.\cite{BFVrounding_Kim}, a subring $\mathcal{R}_t$ of $\mathcal{R}$ is ceated by taking modulo t of rhe ring $\mathcal{R}$. In the case of BFV, we fix this $t$ to the prime integer $p$. The ciphertext also has a modulus q and the message is scaled by a factor $\delta = q/t$.
The plaintext space corresponds to $R_t = \mathbb{Z}[x]/(\Phi_m(x), p)$.
Encryption is then done via following formula:
\begin{align}
\text{Ciphertext:} \quad 
\mathbf{ct} &= \big( \big[\lfloor \Delta \cdot m \rceil + \mathbf{a} \cdot \mathbf{s} + e \big]_q , -\mathbf{a} \big) \\[2mm]
\text{Decryption:} \quad 
m &= \Big\lfloor \frac{c_0 + c_1 \cdot \mathbf{s}}{\Delta} \Big\rceil
\end{align}

And decryption:
The scheme can be implemented as a leveled scheme (SHE) or can be bootstrapped to a fully homomorphic encryption scheme.In BFV, one can perform addition, multiplication and automorphism over the plaintext space.

\subsubsection{Slots}
Smart and Vercauteren \cite{Smart2012FullyHS} noticed that it was possible to encode multiple elements in one plaintext, using the Chinese Remainder Theorem. This splitting in splots can allow SIMD operation - performing a single operation on multiple data. 

BFV can support packing in slots and thus SIMD operations. However, doing this puts a restriction on the use of BFV. If $p$ is the modulus of the plaintext ($\phi_m(b)$), the upper bound of the output noise will grow proportional to to the product of this factor and the sum of the upper bounds on the input noise. When one wants to have a high precision arithmetic, one chooses a high p, which will result in more output noise. This makes SIMD-schemes impractical when performing precise arithmetic calculations, often needed in HE-applications. For instance, privacy-preserving machine learning uses moduli up to 80 bits.\cite{Giland_ML_big_P} Also, a higher value of $p$ enables a higher packing density. The packing density, which is equal to the number of slots divided by the ring dimension, is equal to 1/$d$. $D$ is the multiplicative order of $p$ modulo the cyclotomix index $m$. To achieve full packing, $p$ needs to be larger then $m$. When using power-of-two cyclotomics, the number of slots will be upper bounded by ($p$+1)/2. To conclude, while a large value of $p$ allows for greater packing density and more precise arithmetic operations, it simultaneously exacerbates noise growth.

In CLPX, the idea is to use a plaintext ring modulo $t$, with $t=x-b$ instead of an integer $p$, as in BFV. The plaintext space is now defined as
\begin{equation}
    \mathcal{R}_t = \mathbb{Z}[x]/(\Phi_m(x), x - b) = \mathbb{Z}[x]/(x - b, p) \cong \mathbb{Z}_p
\end{equation}

In this CLPX-scheme, $m$ is a $k$-th power of 2. When encrypting first a hat encoding is performed on the message m, by taking the modulus quotient ring of R modulo t. We get $\hat{m}$ which only has small coefficients. Encryption is done as follows:
\begin{equation}
\mathbf{c} = \Big( \big[ \Delta \cdot \hat{m} + \mathbf{a} \cdot \mathbf{s} + e \big]_q , -\mathbf{a} \Big)
\end{equation}

Decryption is performed using the secret key $\mathbf{s}$:
\begin{equation}
\hat{m} = \Big\lfloor \frac{t}{q} \cdot \big( c_0 + c_1 \cdot \mathbf{s} \big) \Big\rceil
\end{equation}
In CLPX, addition and multiplication can be performed homomorphically.\cite{Chen_CLPX_paper}.

CLPX can encrypt a single huge integer modulo $\phi_m(b)$ and has much lower noise growth when compared to BFV - the growth is sublinear in $b$ (instead of $\phi_m(b)$). This makes CLPX suitable for high-precision arithmetic HE operations. 
However, in CLPX Only a single element is encrypted, so no SIMD operations can be performed. Also, since the size of $p$ is exponential in $m$, there are no known bootstrapping techniques for CLPX. 

\subsection{GBFV}
CLPX has much lower noise growth when compared to BFV, but does not support SIMD operations and is not known to be efficiently bootstrappable for cryptographically secure parameters. Geelen and Vercauteren propose the GBFV scheme which combines the SIMD and bootstrapping capabilities of BFV with the lower noise growth of CLPX, by tuning the parameters $m$ and $t(x)$. Combing both properties would either yield a scheme capable of evaluating deeper circuits or would yield a scheme capable of working with smaller ring dimensions. 

GBFV operates over the cyclotomic ring $\mathcal{R}=\mathbb{Z}[x]/\phi_m(x)$. The plaintext space is defined modulo an arbitrary non-zero principal ideal generated by a polynomial $t=t(x)$. This ring is $R_t=R/tR$. 
A plaintext $m \in \mathcal{R_t}$ is encrypted into a ciphertext $ct \in \mathcal{R_q^2}$ with RLWE:  
\begin{equation}
    ct = \left( \big[ \lfloor \Delta \cdot m \rceil + a \cdot s + e \big]_q, \; -a \right)
    \footnote{$\lfloor x \rceil$ is rounding to the nearest integer}
\end{equation}


The ciphertext space will be a ring $\mathcal{R}_q^2$ with $q\ge2$. The scaling factor $\Delta$ is defined by $q/t$, with $q$ the ciphertext modulus. This scaling factor is not rounded to $\mathcal{R}$, resulting in a conceptually simples scheme definition when compared to BFV and CLPX.

For correct decryption, the canonical infinity norm of the plaintext modulus must be much smaller then the ciphertext modulus. This ensures that the decryption correctly recovers plaintexts without modular wrap-around or rounding errors, all contributions from $t(x)$ (i.e. $t(x)*m(x)$) and the noise much be much smaller than q.

\subsection{Scheme functions}
The GBFV scheme has several functions which it can perform:
\begin{itemize}
    \item Secret key generation: Samples a secret key $s$ from a key distributon $\chi_{key}$, $s \in \mathcal{R}$, returns s.
    \item Relinearization key: after multiplication of ciphertexts, one gets a higher order  polynomial in $s$, which can not be decrypted since the scheme only knows $s$ and not $s$ to a higher power. The relinearization key approximates the ciphertext back to a linear equation in s. Returns the evaluation key.
    \item Decryption: A ciphertext is decrypted using $m = \left\lfloor \frac{c_0 + c_1 \cdot s}{\Delta} \right\rceil$. Returns $m$.
\end{itemize}

GBFV supports standard homomorphic operations on ciphertexts, using following functions:
\begin{itemize}
    \item Ciphertext-ciphertext addition: ciphertext addition is done component-wise modulo $q$ and returns $ct_{add}$. 
    \item Plaintext-ciphertext addition: the plaintext is encrypted as follows: \[ct' = \left( \big[ \lfloor \Delta \cdot m \rceil \big]_q, \; 0 \right)\]
    After this stage, add the original ciphertext with $ct'$ (ciphertext-ciphertext addition).
    \item Key switching: reduces the result of the ciphertext-ciphertext multiplication back to two components.
    \item Ciphertext-ciphertext multiplication: two ciphertexts $ct (c_0,c_1)$ and $ct'=(c_0',c_1')$ are multiplied as follows: 
\begin{align}
\mathbf{c}'' &= \Big( 
  \big[ \big\lfloor \frac{c_0 \cdot c'_0}{\Delta} \big\rceil \big]_q,\;
  \big[ \big\lfloor \frac{c_0 \cdot c'_1 + c_1 \cdot c'_0}{\Delta} \big\rceil \big]_q
\Big), \\[2mm]
c''_2 &= \Big[ \big\lfloor \frac{c_1 \cdot c'_1}{\Delta} \big\rceil \Big]_q
\end{align}
    Since the $c_2''$ contains a second-order term in s, a relinearization is performed using the relinearization key, creating $ct'''$. This ciphertext is then added to $ct''$. 
    \item Ciphertext-plaintext multiplication: takes the ciphertext and multiplies both parts with the flattened message $m$ \footnote{Flattening involves reducing the coefficients from, ensuring the coefficients are reduced modulo $t$ but expressed in $\mathcal{R}$. Following formula is used: 
\[\text{Flatten} : \mathcal{R}_t \rightarrow \mathcal{R}: \quad 
m \mapsto t \cdot \big[ \frac{m}{t} \big]_1
\]}

    \item Automorphism. Applying an automorphism to the ciphertext polynomials, this permutes slots in packed plaintexts after decryption. Then, the plaintext moduli are corrected if they changed under the automorphism. A key switching is performed to bring back the secret key to s. Finally, the adjusted ciphertexts are combined to form the final output. 
    
\subsection{SIMD}


 
\end{itemize}

\section{Private information retrieval}
When retrieving information from a remote server, the database holder will know which elements are queried. To protect the user, one wants to hide which elements are queried from the server. Private information retrieval or PIR is often considered to achieve this goal.

Private information retrieval (PIR) is the process where a user retrieves information from a database without revealing to the database what he is retrieving. The goal is to ensure the server does not learn anything about the index from the user query. 
This will enhance the privacy of the user, since no information will be leaked to the (remote) server(s), potentially causing serious privacy issues. PIR finds application in multiple scenarios where sensitive data are used. For example, a doctor querying a database with patient's medical data will get back the requested medical information, without the server learning which patient or which patient record was requested. PIR also finds applications in technology. Apple uses PIR to provide caller ID information of an incoming phone call, without them learning who is calling who \cite{apple_PIR}. 

 
PIR protocols can be categorized in two groups: single-server PIR and multi-server PIR. 
The single-server PIR is the most straightforward setting: one server holds the full dataset, and the client queries the server to get the data of interest.
In multi-server PIR, there are multiple servers holding a copy of the full dataset, and the client queries multiple servers to obtain the data of interest. The core idea when using multi-server PIR is that, although the dataset is replicated on multiple servers, the query is split into parts. In this way, none of the servers learn which bit is requested but the requested bit can be recovered from the results of the different servers. 

\[
\textbf{Database: } 
D = [\, b_1,\, b_2,\, b_3,\, b_4 \,]
\]
The client wants to retrieve \( b_3 \) privately.

---

\[
\textbf{Step 1: Query generation}
\]
The client samples a random binary vector:
\[
q_1 = [\, q_{11},\, q_{12},\, q_{13},\, q_{14} \,]
\]
and constructs
\[
q_2 = q_1 \oplus e_3
\]
where 
\[
e_3 = [\, 0,\, 0,\, 1,\, 0 \,]
\]
is the unit vector with a 1 in the 3rd position.

The client sends:
\[
q_1 \text{ to Server 1}, \quad q_2 \text{ to Server 2.}
\]

---

\[
\textbf{Step 2: Servers compute answers}
\]
Each server computes the XOR of the database entries where its query has 1s:
\[
a_1 = q_1 \cdot D = \bigoplus_{j=1}^{4} (q_{1j} \cdot b_j)
\]
\[
a_2 = q_2 \cdot D = \bigoplus_{j=1}^{4} (q_{2j} \cdot b_j)
\]

---

\[
\textbf{Step 3: Client combines responses}
\]
\[
a_1 \oplus a_2 
= 
\left( \bigoplus_{j=1}^{4} (q_{1j} \cdot b_j) \right)
\oplus
\left( \bigoplus_{j=1}^{4} (q_{2j} \cdot b_j) \right)
= b_3
\]

---

\[
\boxed{
\text{Client learns } b_3 \text{, and neither server learns which } b_i \text{ was requested.}
}
\]

Security holds as long as the servers do not collude. This assumption is difficult to achieve, because the database has to be on multiple servers, but one party can not have control over these servers. Therefore, single-server PIR schemes are often preferred since they rely on cryptographic hardness assumptions, but at the cost of incurring a huge performance overhead.  
In this thesis, we will further focus on single-server PIR. 

A scheme is information-theoretic secure when the queries asked by the user give no information whatsoever about the requested bit. 
For a single-server PIR scheme, the trivial information-theoretic secure scheme is sending the whole database to the client, he can then query the database and the server will have no information about the selected bit. This gives a communication of O(n). Single-server PIR schemes with smaller communication cost are computationally secure, not information-theoretic secure.   \cite{Christien1999} \cite{kushilevitz1997} \cite{sion2007}
Kushilevitz and Ostrovsky made use of the number-theoretic assumption to deduce a single-server computationally secure PIR with subpolynomial communication. The scheme has a communication complexity of O(n to the epsilonth) for any epsilon > 0. This scheme however requires n big integer multiplications.
Cachin et al. proposed a two-round computationally secure PIR using the phi-hiding assumption where communication complexity is polylogarithmic in n. This scheme requires n modular exponentiations, with large moduli, which makes multiplication slower then in Kuhilevitz's scheme.
Chang subsequently proposed a scheme with logarithmic communication complexity, using Paillier's cryptosystem.
As Sion and Carbunar pointed out, these single-server PIR protocols are mostly orders of magnitude slower than the trivial transfer of the entire database to the client.

FHE-based PIR allows computation over encrypted data and offers optimal communication and computation complexity. 

\section{Feanor-math and feanor}
Feanor-math and feanor are bothe rust libraries made by Simon Pohmann a PhD student at Royal Holloway, University of London. Hise filed of studie is criptographie and computational mathematics. \cite{feanor_creator} Feanor-mat is a library for number theory, and feanor is a library that provides implementions of building blocks for HE, build on feanor-math.

\subsection{Feanor-math}
Like mentiond before feanor-math is a library for number theory. The librari is complietly rithen in rust. The library starts from a main trait\footnote{Trait definitions are a way to group method signatures together to define a set of behaviors necessary to accomplish some purpose.\cite{rustbook}} \texttt{Ring}, and then creathes a thread yiarchie for additional properties.
\chapter{PIR implementation}
\label{cha:PIR_implementation}

This chapter discusses the implementation of different PIR protocols. First, some utility functions that are used in the implementations are explained. Next, the implementation of the PIRANA protocol for small and large payloads is presented. After that, an alternative implementation using one-hot encoding is discussed. Finally, the testbench used to benchmark the different implementations with large payloads is explained. 

The implementations can be found in the thesis github\footnote{\url{https://github.com/antoinejvdm/easygbfv_PIR}}. The examples in the gihub from 4 to 7 are created to test and debug the different implementations. Therefore, a significant amount of steps are printed and the user can choose which element is queried from the database at runtime. This is not suitable for benchmarking. Therefore, examples 8 and 9 are created to benchmark PIRANA and one-hot encoding for large payloads. These implementations do not print unncessary information, the queried element is chosen at random and timing measurements are performed in order to benchmark the implementations.

\section{Utility functions}
Some utility functions are created to help with the implementation of the PIR protocols. These functions are implemented in \verb|util.rs|.

\begin{itemize}
    \item \verb|GenFromI32(&Ring, &[i32])|: Generates a vector of ring elements, for the specified ring, from a vector of i32 values.
    \item \verb|GenFromBigInt(&Ring, &[BigInt])|: Generates a vector of ring elements, for the specified ring, from a vector of BigInt values.
    \item \verb|d3_finder(element_size_bit: usize, p_mod: &str)|: This function will determine in how many chunks a large payload can be split. It will return the amount of chunks. To achieve this, the function needs to know the integer modulus \texttt{p\_mod} and the maximal size of an element in the database (in bits). 
    
    Equation \ref{eq:chunk_estimator} shows how the number of chunks are calculated. 
    \begin{equation}
        \text{number of chunks} = \left \lceil \frac{\texttt{element\_size\_bit}}{\lfloor log_2(\texttt{p\_mod}) \rfloor} \right \rceil
        \label{eq:chunk_estimator}
    \end{equation}
    \item \verb|calculate_cw_len(k,columns)|: The constant-weight codeword length $m$ is calculated with this function. The function takes the Hamming weight $k$ and the amount of columns in the database as input arguments. If $k$ equals 2 (default parameter), the function calculates $m$ via this formula: 
    \begin{equation}
        \frac{m \cdot (m-1)}{2} \ge n_\text{col}
        \; \Longrightarrow\;
        m=\left\lceil\frac{1+\sqrt{1+8\cdot n_\text{col}}}{2}\right\rceil
    \end{equation}.
    In any other case, $m$ will be calaculated via a iterative approach. 
    It will return the minimal $m$ such that $\binom{m}{k} \geq columns$.
    \item \verb|base_p_decompose(n,p, chunks)|: This function will do a base-p decomposition of a big integer $n$ into chunks. Euclidian division of the integer $n$ by the plaintext modulus $p$ is used. The division will be done $chunks$ amount of times, and every chunk will keep the remainder of the division. This will create a vector of size $chunks$, containing numbers smaller than $p$.
    \item \verb|recompose_base_p_to_str(digits, p)|: This function will recompose the chunks back into one large integer. It will take the vector of chunks and the plaintext modulus $p$. The recomposition is done by multiplying every chunk with $p^i$, with $i$ the index of the chunk in the vector. The results are summed together to create one large integer, which is then converted to a string and returned.
    \begin{equation}
        n = \sum_{i=0}^{chunks-1} digits[i] \cdot p^i
    \end{equation}
    \item \verb|get_rand_matrix(nr_elements, element_size_bits, nr_slots, p)|: To create the database, this function is used. It will create a 3D-matrix with size $r \times t \times chunks$, with $r$ the number of slots in a ciphertext, $t$ the number of elements divided by the number of slots, and $chunks$ the amount of chunks needed to split one large element. Every element in the database is a large integer with size equal to $element\_size\_bits$. Every large integer is split into $chunks$ chunks via base-p decomposition. This function will return the 3D-matrix. To avoid recreating the matrix every time the function is called upon, the matrix is stored in a cache file. If the cache file already exists, the matrix is loaded from the file instead of being created again.
    % To increase the performance, one can already shift the elements of the next chunks. This will allow to avoid the rotate-operation when the server is computing, to fit multiple chunks in one ciphertext.  
\end{itemize}



\section{GBFV-PIRANA, single-query small payload}
In this section, the implementation of the PIRANA protocol using the easyGBFV library is presented. The implementation is a single-query implementation for small payloads. This means that the elements of the database are smaller when compared to the plaintext modulus $p$. The single-query small payload implementation can be found in the \verb|examples\5_GBFV_PIRANA_Spayload| folder of the thesis github.

First, a database/matrix is created with size $r \cdot c$, where $r$ is equal to the number of slots in the ciphertext and $c$ is equal to the amount of elements divided by the number of slots in the ciphertext\footnote{When working with small payloads, all elements in the database are of maximal size i32 or plaintext modulo.}. All indices of the columns of the matrix are substituted with a constant weight codeword. To achieve this, $m$ and $k$ have to be chosen properly. In this implementation, $k$ will be set to 2, meaning that every codeword has a Hamming weight of 2. This will keep the amount of ciphertext-ciphertext multiplications low. Knowing $k$, $m$ can be calculated as $c \leq \binom{m}{2}$, with $c$ the amount of columns in the matrix. Later, every column will be multiplied with a ciphertext. Therefore, the plaintext elements of one column are set into a plaintext ring.

Subsequently, an instance of GBFV is created. When creating this instance, one has to set $m$ the cyclotomic order, $p$ the integer modulus and $t$ the plaintext modulus. EasyGBFV has some GBFV parameters already set, to create GBFV instances of 16/32/64 bits of plaintext modulus. 

Having a database and having created a GBFV instance, the PIRANA set-up is finished. The client can now create a query for an element in the database. Imagine the client wants to retrieve element ($i,j$) from the database. First, the client will look up which codeword corresponds to column $j$. The client will create the query matrix, which is a matrix of size $r \cdot m$. This matrix is an all-zero matrix, except for the $i$-th row, which is subsituted with the codeword. 

Before sending the query, the client has to encrypt the query. Therefore, he generates a secret key and a public key. Every column of length $r$ (amount of slots in a ciphertext) will be encrypted. The client sends $m$ ciphertexts to the server. 

The server will, for each column, take the codewords of length $m$ and look at which position the codeword has a 1. In our case, there are only two one's (remember, the Hamming weight equals 2). The server will take the corresponding ciphertexts of these two positions in the query and multiply them with each other. This new ciphertext is one column of the selection matrix. This process is repeated for all $c$ columns of the database. After creating the selection matrix, the server will perform a homomorphic plaintext-ciphertext multiplication between every column of the selection matrix and the corresponding column of the database. Finally, all the columns of the resulting matrix are summed together, by going through all the columns and adding them via an accumulator. The result is then sent back as one ciphertext to the client. 

The client will receive the ciphertext from the server and will decrypt using his secret key. Every ciphertext is decrypted and will return a vector of slot ring elements. All elements are equal to zero, except for the $i$-th element, which is equal to the desired element in the database. The client can now format this element and retrieve the desired value. 

\section{GBFV-PIRANA, single-query large payload}
In this section, the implementation of a single-query PIR protocol for large payloads using the easyGBFV library is presented. The implementation can be found in the \verb|examples\7_GBFV_PIRANA_Lpayload| folder of the thesis github. 
The implementation is similar to the small payload implementation, with some differences. First, the database is created. The database is now a 3D-matrix of size $r \cdot t \cdot chunks$, where $r$ is equal to the number of slots in the ciphertext, $t$ is equal to the amount of elements divided by the number of slots and $chunks$ is equal to the amount of chunks needed to split one large element. Every large element in the database is split into $chunks$ chunks via base-$p$ decomposition, with $p$ the plaintext modulus. Every chunk is smaller than $p$. The query generation works in the same way as for small payloads, where a constant-weight codeword is set at the $i$-th row of the query matrix. At the server side, the implementation works in the same way as the small payload implementation. The selection matrix is created in the same way, but when multiplying the selection matrix with the database, this operation is repeated for every chunk of the database. This will result in $chunks$ ciphertexts. To reduce the number of ciphertexts sent back to the client, a rotate-and-sum operation is performed on every ciphertext. This will result in $\frac{chunks}{slots}$ ciphertexts which are sent back to the client as shown in Algorithm \ref{al:rot_and_sum}. The client will decrypt every ciphertext and recompose the chunks into one large integer via base-$p$ recomposition. The client can now retrieve the desired element from the database.

\begin{algorithm}
\caption{Rotate-and-sum ($chunks \leq slots$)}\label{al:rot_and_sum}
\begin{algorithmic}[1]
\Require Ciphertext list $ct\_elements$, number of slots $slots$
\Ensure $return\_query$ where every slot contains one $ch$

\State Initialize empty list $return\_query$
\State $i \gets 0$
\State $acc \gets ct\_elements[i]$
\State $i \mathrel{+}= 1$
\While{$i < |ct\_elements|$}
    \State $acc \gets \text{Rotate}(acc, 1)$ \Comment {Rotate by 1 position}
    \State $acc \gets \text{homAdd}(acc, ct\_elements[i])$
    \State Append $acc$ to $return\_query$
    \State $i \mathrel{+}= 1$
\EndWhile

\State \Return $return\_query$
\end{algorithmic}
\end{algorithm}



\section{GBFV one-hot encoding}
As shown in the PIRANA paper \cite{PIRANA2023}, PIRANA is slower when compared to most other PIR protocols when querying one element. PIRANA becomes more competititive when querying multiple elements at once. Therefore, using the PIRANA protocol to get one element is suboptimal. An alternative way to retrieve one element from a database is to use one-hot encoding. The implementation can be found in the \verb|example\6_OneHot_Lpayload|.
Instead of sending a query matrix, as in PIRANA, with one-hot encoding only two vectors are sent. Both vectors are all-zero vectors, except at position $i$ for the first vector (row vector) and at position $j$ for the second vector (column vector). When using small elements, the database in one-hot encoding has to be structured in a 2D-matrix. The first dimension equals the number of slots, while the second dimension equals $t$, where $t = \frac{n}{s}$ with $n$ the amount of elements in the database and $s$ the number of slots in a ciphertext. Algorithm \ref{al:OH_client} shows the query generation for large or small payloads. The matrx multiplication used in Algorithm \ref{al:OH_server} needs a vector of length equal to a multiple of the number of slots. Therefore, padding might be needed, when the amount of columns is not a multiple of the number of slots.

% one hot encoding query generation
\begin{algorithm}
\caption{One-hot query generation} \label{al:OH_client}
\begin{algorithmic}[1] % [1] adds line numbers

\State $columns \gets \frac{elements}{slots}$ 

\If{$columns \bmod slots \neq 0$}
    \State $padded\_columns \gets columns + (slots - (columns \bmod slots))$
\EndIf

\vspace{0.3em}

\State $j \in \{0, \dots, columns - 1\}$
\State $i \in \{0, \dots, slots - 1\}$

\vspace{0.3em}

\State $col\_selector \gets$ array of zeros of length $padded\_columns$
\State $elem\_selector \gets$ array of zeros of length $slots$
\State $col\_selector[j] \gets 1$
\State $elem\_selector[i] \gets 1$

\vspace{0.3em}

\State $row\_selector\_slots \gets \texttt{GenFromI32}(slotring, elem\_selector)$
\State $col\_selector\_slots \gets \texttt{GenFromI32}(slotring, col\_selector)$

\end{algorithmic}
\end{algorithm}
The two query vectors are encrypted and sent to the server. The server will perform a matrix multiplication between the column vector and the database, resulting in a ciphertext containing only the $j$-th column of the database. This ciphertext is then multiplied with the row vector, resulting in a ciphertext containing only the desired element. This ciphertext is sent back to the client, who can decrypt and retrieve the desired element. When handling large elements, the database is set up as a 3D-matrix, where the third dimension equals the amount of chunks needed to split one large element. The server will need to perform the matrix multiplication for every chunk, resulting in multiple ciphertexts. Algorithm \ref{al:OH_server} shows the server-side operations for both large payloads. 
% one hot encoding server side
\begin{algorithm}
\caption{One-hot encoding search algorithm}
\label{al:OH_server}
\begin{algorithmic}[1]
\State $nrChunks \gets$ number of chunks per element
\For{$ch = 0$ \textbf{to} $numChunks-1$}

    \State $layer \gets$ empty 2D array with capacity $numChunks \times slots$
    \State $colInChunk \gets$ number of columns in one $ch$ of the matrix

    \For{$r = 0$ \textbf{to} $slots-1$}
        \State $rowVals \gets$ empty array of length $paddedColumns$ 

        \For{$c = 0$ \textbf{to} $paddedColumns-1$}
            \If{$c < colInChunk$}
                \State Append $matrix[ch][r][c]$ to $rowVals$
            \Else
                \State Append $0$ (zero element in ring) to $rowVals$
            \EndIf
        \EndFor

        \State $slotsRow \gets \textsc{GenFromBigInt}(slotring, rowVals)$ %slotring is the plaintext slot ring of GBFV}
        \State Append $slotsRow$ to $layer$
    \EndFor

    \State $data \gets$ empty array with capacity $slots \times paddedColumns \times paddedColumns$
    \For{$rowIdx = 0$ \textbf{to} number of rows in $layer -1$}
        \For{$colIdx = 0$ \textbf{to} $paddedColumns-1$}
            \State Append $layer[rowIdx][colIdx]$ to $data$
        \EndFor
    \EndFor

    \State $mm \gets \textsc{DenseMatrixMul}(slotring, padded\_columns, data)$

    \State $ct\_col \gets \textsc{HomMatMul}(gbfv, mm, ct\_col\_select, pk)$

    \State $ct\_chunk\_element \gets$ element-wise homomorphic multiplication of $ct\_col$ and $ct\_row\_select$ using $gbfv$

    \State Append $ct\_chunk\_element$ to $ct\_elements$

\EndFor
\end{algorithmic}
\end{algorithm}

A rotate-and-sum operation is performed to reduce the amount of ciphertexts sent back to the client. Communication cost can be reduced when using one-hot encoding instead of PIRANA, as discussed in Chapter \ref{cha:results}.

\section{Testbench}
To benchmark the different implementations, two testbenches are created. The first testbench is testing PIRANA for large payloads while the second is testing one-hot encoding for large payloads. All operations performed by the server are set in a seperate Rust file. This file will have one function that mimics the working of the server. This function takes the GBFV instance, the private key, the encrypted queries and the database as input arguments. The output of this function will return the selected element as a ciphertext. 
Both testbenches work in a similar manner. They start by making the GBFV instance. When running the testbench, the user can pass arguments to choose the size of the plaintext modulus and the $N$, the cyclotomic order. The creation of the instance is timed. Next, the database is created. The user can pass arguments to choose the amount of elements in the database and the size of each element (in bits). The database is processed. For GBFV every column of \texttt{BigInt} is converted into a slotted plaintext, that can be used for ciphertext-plaintext multiplications. Algorithm \ref{al:db_process_PIRANA} shows the database processing for PIRANA. The original matrix contains all \texttt{BigInt} elements of the database, split into chunks. The algorithm returns a processed matrix, which is a 2D-matrix of size $chunks \times columns$, where every element is a slotted plaintext. 
For one-hot encoding every chunk is set into a dense \texttt{DenseMatrixMul}, which is a type which is used for the matrix multiplication. The database creation is also timed.  
In order to produce realistic measurements of the performance of the server, one element is chosen at random and is queried from the database. This is done multiple times and the average time is taken. Every iteration, the query generation time, the server response time and the decryption and recomposition time are measured. The testbenches will print the average times of every operation. Next to these three timing measurements, every time an addition, multiplication (ct-ct, pt-ct) and rotation is performed by the server, a counter is increased. The time spent for each of these specific operations is also measured. At the end of the testbench, the total amount of operations and the average time spent per operation is printed. This will give insight into which operations are the most costly and where future optimizations can be made. 
At the end of every iteration, a check is done to verify the correctness of the retrieved element. This retrieved element is compared to the original element in the database.

% sata prossesing server side PIRANA
\begin{algorithm}
\caption{Processing of database for PIRANA}
\label{al:db_process_PIRANA}
\begin{algorithmic}[1]
\State $chunks \gets$ number of chunks per element
\State $colums \gets \frac{elements}{slots}$
\State $matrix \gets$ 3D-matrix of size $slots \times columns \times chunks$
\State $ProcessedMatrix \gets$ empty matrix with capacity $chunks \times colums$
\For{$ch = 0$ \textbf{to} $chunks-1$}

    \State $chRows \gets$ empty vector with capacity $colums$

    \For{$c = 0$ \textbf{to} $columns-1$}
        \State $rowVals \gets$ empty array of length $slots$ 
        
        \For{$r = 0$ \textbf{to} $slots-1$}
            \If{$c < colInChunk$}
                \State Append $rowVals$ with $matrix[ch][r][c]$
            \Else
                \State Append $rowVals$ with $0$ (zero element in ring)
            \EndIf
        \EndFor

        \State $slotsRow \gets \textsc{GenFromBigInt}(slotring, rowVals)$ %slotring is the plaintext slot ring of GBFV}
        \State $slotsRowPt \gets \textsc{PlaintextFromSlots}(slotsRow)$ 

        \State Append $chRows$ with $slotsRowPt$
    \EndFor

    \State $ProcessedMatrix[ch] \gets chRows$

\EndFor
\end{algorithmic}
\end{algorithm}

 



\chapter{Results}
\label{cha:results}

\section{One-hot encoding}

\subsection{Communication cost}
One-hot encoding can be used to reduce the communication cost when querying one element from a database.
The communication cost in PIRANA when querying one element is $m$ ciphertexts, where $m$ is equal to $O\!\left(\sqrt[k]{k! \, n}  + k \right)$. The communication cost in one-hot encoding is equal to one ciphertext for the rows, because there are as many rows as slots, and $t / s$ ciphertexts for the columns, since there are more columns than slots in a ciphertext. So the query communication cost, which is the amount of ciphertexts sent from client to server, is equal to:  
\begin{equation}
    \texttt{query communication cost} = 1 + \frac{n}{s^2}
\end{equation}

As can be deduced from the formulas above, the communication cost in one-hot encoding will be lower for small databases (small $n$). When increasing the number of slots, the communication cost will be lower in one-hot encoding up to a larger database size. For example, when taking a Hamming weight of 2 for PIRANA and the amount of slots equal to 16, one-hot encoding has a lower communication cost up to a databse size of 131583 elements. When lowering the number of slot to 1, one-hot encoding only has a lower communication cost for a database up to 3 elements. From that moment on, PIRANA will have a lower communication cost.

\begin{table}[h]
    \centering
    \begin{tabular}{r|rrrrrrr}
        Slots $s$ & 2 & 4 & 8 & 16 & 32 & 64 & 128 \\
        \hline
        Elements $n$ 
        & 46 & 574 & 8446 & 132094 & 2101246 & 33570814 & 536936446 \\
        Query ct's
        & 11 & 35 & 131 & 515 & 2052 & 8196 & 32772 \\
    \end{tabular}
    \caption{Number of elements $n$ from when PIRANA has less communication cost than one-hot encoding (results are indicated for  different amount of slots $s$ and Hamming weight $k=2$).}
\end{table}

\subsection{One-hot encoding versus GBFV}


\section{GBFV-PIRANA versus BFV-PIRANA}
\subsection{Large payload, $N=2^{13}$, compared to paper}
\subsection{Large payload, compared to our implementation}

\section{GBFV PIRANA, best parameters}
\chapter{Discussion}
\label{cha:Discussion}

\section{Discussion of results}
Discussion of results goes here.
\section{Why GBFV is better than BFV}
Discussion of why GBFV is better than BFV goes here.

\section{Future work}
Discussion of future work goes here.
\chapter{Conclusion}
\label{cha:conclusion}
In this thesis, PIRANA is for a first time successfully implemented with a GBFV scheme.
To this end, GBFV-PIRANA was implemented on top of the Fheanor library using the easyGBFV wrapper. Both single-query PIRANA variants, targeting small and large payloads, were implemented. A comprehensive testbench was developed to evaluate performance of the developed implementation for different database sizes, payload sizes, and plaintext moduli.

This is of particular interest, since GBFV allows for greater flexibility in parameter selection; fewer but bigger slots are possible when compared to BFV (cf. PIRANA was implemented with BFV by Liu et al. \cite{PIRANA2023}). Also, GBFV could allow to evaluate deeper circuits or could allow to work with smaller ring dimensons, by combining the properties of BFV and CLPX (Section \ref{sec:GBFVtheor}). By decoupling the number of slots from the ring dimension and allowing larger plaintext moduli, GBFV can accommodate larger payload chunks and reduce the number of ciphertexts required for large-payload retrieval. This can lead to lower communication overhead. 

The experimental results demonstrate that GBFV-PIRANA offers clear advantages over one-hot encoding in terms of communication efficiency once the database reaches a certain size threshold, even for single-query PIRANA. This threshold depends on the number of SIMD slots and the chosen constant-weight code parameters, but aligns well with theoretical expectations.

Experimental results show that implementing GBFV-PIRANA is feasible, but performance lags when compared to the original PIRANA-paper implemented on a performant library. This is mainly due to the fact that Fheanor is not optimized for performance, but rather for research purposes. This opens directions for future work, such as implementing GBFV-PIRANA on a more performant library and implementing multi-query GBFV-PIRANA thereon. 

This thesis showed the feasibility of implemeting PIRANA with a GBFV scheme and provided insights into its performance characteristics.




% % If you have appendices:
% \appendixpage*          % if wanted
% \appendix
% \include{appendixes/appendix-1}
% % ... and so on until
% \include{appendixes/appendix-n}

\backmatter
% The bibliography comes after the appendices.
% You can replace the standard "abbrv" bibliography style by another one.
\bibliographystyle{abbrv}
\bibliography{references}

\end{document}
